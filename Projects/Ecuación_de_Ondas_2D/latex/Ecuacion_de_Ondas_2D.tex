\documentclass[10pt]{article}
\usepackage[latin1]{inputenc}
\usepackage[spanish]{babel}                                 						     % language
	\spanishdecimal{.}
\usepackage{graphicx}
\usepackage{amsmath, amsfonts, amssymb}                     						     % amspackages
\usepackage{color}                                          						     % color
\usepackage[left=1.4cm,right=1.4cm,top=1.4cm,bottom=1.8cm, paperwidth=216mm, paperheight=330mm]{geometry}
\usepackage{enumitem}
\usepackage{multicol}
\title{\textbf{PEP 1}}
\author{Omar Fern�ndez -- omar.fernandez.o@usach.cl}

\usepackage{fancyhdr}
\lhead[]{}
\chead[]{}
\rhead[]{}
\renewcommand{\headrulewidth}{0pt}

\lfoot[]{}
\cfoot[]{\thepage}
\rfoot[]{}
\renewcommand{\footrulewidth}{0pt}
\pagestyle{fancy}

\fancypagestyle{firstpage}{
\fancyhead[L]{}
\fancyhead[C]{}
\fancyhead[R]{}
\fancyfoot[L]{}
\fancyfoot[C]{\thepage}
\fancyfoot[R]{}
\renewcommand{\footrulewidth}{0pt}
\renewcommand{\headrulewidth}{0pt}
}

% -- New commands and Renew commands
%\renewcommand{\labelitemi}{$\blacktriangleright$}
%\renewcommand{\labelitemii}{$\blacktriangleright$}
%\renewcommand{\thesubfigure}{\arabic{figure}.\alph{subfigure}}
%\renewcommand{\bibfont}{\fontsize{8}{8.5}\selectfont}
\newcommand{\HRule}[1]{\rule{\linewidth}{#1}}
\newcommand{\figref}[1]{\textbf{\textsf{Fig. \ref{#1}}}}
\newcommand{\tabref}[1]{\textbf{\textsf{Tabla \ref{#1}}}}
\newcommand{\subfigref}[1]{\textbf{\textsf{Fig. \subref{#1}}}}


\newcommand{\pt}[1]{\textbf{\textsf{#1}}}


% -- New enviroments and Renew enviroments
\newenvironment{logotype}[1]{\vspace{-1cm}#1}{}
\newenvironment{titulo}{ \centering \Large \bfseries \sffamily}{ \\  \vspace{3pt}}
\newenvironment{authorinfo}[7]{\centering \textsf{\textbf{#5}} \par Profesor: \textsl{#1} -- \textcolor{blue}{#2}\par Ayudante: \textsl{#3} -- \textcolor{blue}{#4}\par \textit{#6}\par #7\par}{}
\newenvironment{resumen}{\footnotesize \begin{center} \textbf{Resumen} \end{center} \vspace{-7pt}}{\vspace{3pt}}
\newenvironment{keywords}{\footnotesize \textbf{Palabras Claves:}\vspace{-7pt}}{\vspace{3pt}}

% -- Counter defintion
\newcounter{problem}


\newenvironment{problem}[1]
  {\noindent\stepcounter{problem}\large\textbf{\textsf{\textcolor{black}{Problema \theproblem. #1}}} \normalsize }
  {}

% -- Setlength fixing
\setlength{\parindent}{0pt}
\setlength{\parskip}{5pt plus 2pt minus 1pt}

\begin{document}
\thispagestyle{firstpage}

%\twocolumn[
\begin{logotype}
  {\includegraphics[scale=0.08]{USACHcolor.pdf}}
\end{logotype}

\vspace*{-1cm}

\begin{titulo}
Proyectos -- F�sica Computacional IV
\end{titulo}

\vspace*{0.2cm}

\begin{authorinfo}
  {Omar Fern�ndez Olgu�n}
  {omar.fernandez.o@usach.cl}
  {Nicol�s Campos Agusto}
  {nicolas.campos.a@usach.cl}
  {Astrof�sica con menci�n en ciencia de datos}
  {Departamento de F�sica, Facultad de Ciencias, Universidad de Santiago de Chile}
  {\today}
\end{authorinfo} 


\vspace*{0.2cm}

\vspace{1ex}\hrule\vspace{1ex}



\noindent \large{\textbf{\textsf{Resoluci�n de la Ecuaci�n de Ondas en 2D mediante Diferencias Finitas}}} \normalsize

El objetivo de este trabajo es modelar la propagaci�n de ondas en un medio bidimensional utilizando el m�todo de diferencias finitas para resolver la ecuaci�n de ondas dependiente del tiempo. Se estudiar� la propagaci�n de ondas radiales generadas por una y/o m�ltiples fuentes, as� como los fen�menos de interferencia que emergen de la superposici�n de frentes de onda. Se analizar�n distintas condiciones iniciales y de frontera, junto con la estabilidad num�rica del esquema mediante el criterio de Von Neumann. Los resultados se presentar�n mediante visualizaciones 2D y 3D y, en lo posible, animaciones.

\vspace{0.5cm}

\noindent \large{\textbf{\textsf{An�lisis Principales:}}} \normalsize

\begin{enumerate}

\item \textbf{Ecuaci�n de ondas en dos dimensiones}

La ecuaci�n de ondas en un medio bidimensional se escribe como
\[
\frac{\partial^2 u}{\partial t^2}
=
c^2 \left(
\frac{\partial^2 u}{\partial x^2}
+
\frac{\partial^2 u}{\partial y^2}
\right),
\]
donde \(u(x,y,t)\) representa el campo de desplazamiento (o amplitud de la onda) y \(c\) es la velocidad de propagaci�n de la onda en el medio.

Utilice diferencias finitas centrales para aproximar las derivadas espaciales y temporales, y proponga el esquema num�rico expl�cito correspondiente.

\vspace{0.2cm}

\item \textbf{Ondas radiales}

Considere condiciones iniciales con simetr�a radial, por ejemplo un pulso gaussiano centrado en el dominio:
\[
u(x,y,0) = A \exp\left( -\frac{(x-x_0)^2 + (y-y_0)^2}{2\sigma^2} \right),
\quad
\frac{\partial u}{\partial t}(x,y,0) = 0.
\]
Analice la propagaci�n radial del frente de onda y su dependencia con la velocidad de propagaci�n \(c\).

\vspace{0.2cm}

\item \textbf{Interferencia de ondas}

Introduzca dos o m�s fuentes puntuales oscilantes ubicadas en distintas posiciones del plano y estudie los patrones de interferencia generados por la superposici�n de ondas. Analice la estructura espacial de m�ximos y m�nimos de interferencia y su evoluci�n temporal, extrayendo un perfil espacial.

\vspace{0.2cm}

\item \textbf{An�lisis de estabilidad de Von Neumann}

Realice un an�lisis de estabilidad tipo Von Neumann para el esquema num�rico en dos dimensiones. Determine la condici�n que deben satisfacer los pasos espaciales \(\Delta x\), \(\Delta y\) y el paso temporal \(\Delta t\) para asegurar la estabilidad del m�todo.


\vspace{0.2cm}

\item \textbf{Visualizaci�n avanzada}

Cree una animaci�n que muestre la propagaci�n de las ondas en el plano y la formaci�n de patrones de interferencia a lo largo del tiempo.

\end{enumerate}

\end{document}
