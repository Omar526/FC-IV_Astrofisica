\documentclass[10pt]{article}
\usepackage[latin1]{inputenc}
\usepackage[spanish]{babel}                                  % language
\spanishdecimal{.}
\usepackage{graphicx}
\usepackage{amsmath, amsfonts, amssymb}                      % amspackages
\usepackage{color}                                          % color
\usepackage[left=1.4cm,right=1.4cm,top=1.4cm,bottom=1.8cm,
            paperwidth=216mm, paperheight=330mm]{geometry}
\usepackage{enumitem}
\usepackage{multicol}
\usepackage[hang,flushmargin]{footmisc}

\title{\textbf{PEP 2}}
\author{Omar Fern�ndez -- omar.fernandez.o@usach.cl}

\usepackage{listings}
\usepackage{xcolor}

\newcommand{\code}[1]{%
  \colorbox{gray!15}{\texttt{#1}}%
}

\usepackage{url}

\usepackage[
    colorlinks=true,
    linkcolor=blue,
    citecolor=blue,
    urlcolor=blue
]{hyperref}

\usepackage{fancyhdr}
\lhead[]{}
\chead[]{}
\rhead[]{}
\renewcommand{\headrulewidth}{0pt}

\lfoot[]{}
\cfoot[]{\thepage}
\rfoot[]{}
\renewcommand{\footrulewidth}{0pt}
\pagestyle{fancy}

\fancypagestyle{firstpage}{
\fancyhead[L]{}
\fancyhead[C]{}
\fancyhead[R]{}
\fancyfoot[L]{}
\fancyfoot[C]{\thepage}
\fancyfoot[R]{}
\renewcommand{\footrulewidth}{0pt}
\renewcommand{\headrulewidth}{0pt}
}

\newcommand{\HRule}[1]{\rule{\linewidth}{#1}}
\newcommand{\figref}[1]{\textbf{\textsf{Fig. \ref{#1}}}}
\newcommand{\tabref}[1]{\textbf{\textsf{Tabla \ref{#1}}}}
\newcommand{\subfigref}[1]{\textbf{\textsf{Fig. \subref{#1}}}}

\newenvironment{logotype}[1]{\vspace{-1cm}#1}{}
\newenvironment{titulo}{ \centering \Large \bfseries \sffamily}{ \\  \vspace{3pt}}
\newenvironment{authorinfo}[7]{%
\centering \textsf{\textbf{#5}} \par
Profesor: \textsl{#1} -- \textcolor{blue}{#2}\par
Ayudante: \textsl{#3} -- \textcolor{blue}{#4}\par
\textit{#6}\par #7\par}{}

\setlength{\parindent}{0pt}
\setlength{\parskip}{5pt plus 2pt minus 1pt}

\begin{document}
\thispagestyle{firstpage}

\begin{logotype}
  {\includegraphics[scale=0.08]{USACHcolor.pdf}}
\end{logotype}

\vspace*{-1cm}

\begin{titulo}
Proyectos -- F�sica Computacional IV
\end{titulo}

\vspace*{0.2cm}

\begin{authorinfo}
  {Omar Fern�ndez Olgu�n}
  {omar.fernandez.o@usach.cl}
  {Nicol�s Campos Agusto}
  {nicolas.campos.a@usach.cl}
  {Astrof�sica con menci�n en ciencia de datos}
  {Departamento de F�sica, Facultad de Ciencias, Universidad de Santiago de Chile}
  {\today}
\end{authorinfo} 

\vspace*{0.2cm}
\vspace{1ex}\hrule\vspace{1ex}

\noindent \large{\textbf{\textsf{Modelaci�n Num�rica del Campo El�ctrico mediante la Ecuaci�n de Poisson}}} \normalsize

El objetivo de este proyecto es determinar num�ricamente el \textbf{potencial el�ctrico} generado por distribuciones de carga continuas y, a partir de �l, calcular el \textbf{campo el�ctrico} mediante el gradiente. La ecuaci�n fundamental a resolver es la ecuaci�n de Poisson:

\[
\nabla^2 \phi(\mathbf{r}) = - \frac{\rho(\mathbf{r})}{\varepsilon_0},
\]

donde \(\phi(\mathbf{r})\) es el potencial el�ctrico, \(\rho(\mathbf{r})\) la densidad de carga, y \(\varepsilon_0\) la permitividad del vac�o.

\vspace{0.5cm}

\noindent \large{\textbf{\textsf{An�lisis Principales:}}} \normalsize

\begin{enumerate}

\item \textbf{Discretizaci�n de Poisson}

\begin{itemize}
\item Defina una malla regular en 2D o 3D con paso \(\Delta x, \Delta y, \Delta z\).
\item Aproximaci�n del laplaciano mediante diferencias finitas centradas:
\[
\nabla^2 \phi_{i,j,k} \approx 
\frac{\phi_{i+1,j,k} - 2\phi_{i,j,k} + \phi_{i-1,j,k}}{\Delta x^2} +
\frac{\phi_{i,j+1,k} - 2\phi_{i,j,k} + \phi_{i,j-1,k}}{\Delta y^2} +
\frac{\phi_{i,j,k+1} - 2\phi_{i,j,k} + \phi_{i,j,k-1}}{\Delta z^2}.
\]
\end{itemize}

\vspace{0.2cm}

\item \textbf{Condiciones de frontera}

\begin{itemize}
\item Defina las condiciones de contorno de la regi�n de estudio. Para esferas aisladas, puede usar \(\phi \to 0\) a grandes distancias.
\end{itemize}

\vspace{0.2cm}

\item \textbf{Soluci�n num�rica}

\begin{itemize}
\item Resuelva el sistema lineal despejando \(\phi_{i,j,k}\) de la ecuaci�n discretizada.
\item Itere hasta alcanzar convergencia mediante el m�todo de \textbf{Jacobi} (simple y f�cil de implementar), o m�todos m�s eficientes como \textbf{Gauss-Seidel} o \textbf{SOR}.
\end{itemize}

\vspace{0.2cm}

\item \textbf{C�lculo del campo el�ctrico}

\begin{itemize}
\item Una vez obtenido \(\phi\) en cada punto, el campo el�ctrico se calcula como
\[
\mathbf{E} = -\nabla \phi.
\]
\item Aproximaci�n del gradiente mediante diferencias finitas centradas:
\[
E_x \approx -\frac{\phi_{i+1,j,k}-\phi_{i-1,j,k}}{2\Delta x}, \quad
E_y \approx -\frac{\phi_{i,j+1,k}-\phi_{i,j-1,k}}{2\Delta y}, \quad
E_z \approx -\frac{\phi_{i,j,k+1}-\phi_{i,j,k-1}}{2\Delta z}.
\]
\end{itemize}

\vspace{0.2cm}

\item \textbf{Visualizaci�n}

\begin{itemize}
\item Grafique l�neas de flujo del campo el�ctrico en 2D o 3D.
\item Visualice mapas de intensidad \(|\mathbf{E}|\) o del potencial \(\phi\).
\item Compare configuraciones: esferas s�lidas vs huecas, distribuciones m�ltiples, densidades no uniformes.
\end{itemize}

\vspace{0.2cm}

\item \textbf{Extensi�n}

\begin{itemize}
\item Resuelva la ecuaci�n utilizando coordenadas esf�ricas para el caso de una esfera cargada y compare con la soluci�n en coordenadas cartesianas.
\item Explore la eficiencia de distintos m�todos iterativos y el efecto del tama�o de la malla.
\end{itemize}

\end{enumerate}

\end{document}
