\documentclass[10pt]{article}
\usepackage[latin1]{inputenc}
\usepackage[spanish]{babel}
\spanishdecimal{.}
\usepackage{graphicx}
\usepackage{amsmath, amsfonts, amssymb}
\usepackage{color}
\usepackage[left=1.4cm,right=1.4cm,top=1.4cm,bottom=1.8cm,
            paperwidth=216mm, paperheight=330mm]{geometry}
\usepackage{enumitem}
\usepackage{multicol}
\usepackage[hang,flushmargin]{footmisc}

\title{\textbf{PEP 3}}
\author{Omar Fern�ndez -- omar.fernandez.o@usach.cl}

\usepackage{listings}
\usepackage{xcolor}
\newcommand{\code}[1]{\colorbox{gray!15}{\texttt{#1}}}

\usepackage{url}
\usepackage[
    colorlinks=true,
    linkcolor=blue,
    citecolor=blue,
    urlcolor=blue
]{hyperref}

\usepackage{fancyhdr}
\lhead[]{} \chead[]{} \rhead[]{}
\renewcommand{\headrulewidth}{0pt}
\lfoot[]{} \cfoot[]{\thepage} \rfoot[]{}
\renewcommand{\footrulewidth}{0pt}
\pagestyle{fancy}

\fancypagestyle{firstpage}{
\fancyhead[L]{} \fancyhead[C]{} \fancyhead[R]{}
\fancyfoot[L]{} \fancyfoot[C]{\thepage} \fancyfoot[R]{}
\renewcommand{\footrulewidth}{0pt} \renewcommand{\headrulewidth}{0pt}
}

\newenvironment{logotype}[1]{\vspace{-1cm}#1}{}
\newenvironment{titulo}{ \centering \Large \bfseries \sffamily}{ \\  \vspace{3pt}}
\newenvironment{authorinfo}[7]{%
\centering \textsf{\textbf{#5}} \par
Profesor: \textsl{#1} -- \textcolor{blue}{#2}\par
Ayudante: \textsl{#3} -- \textcolor{blue}{#4}\par
\textit{#6}\par #7\par}{}

\setlength{\parindent}{0pt}
\setlength{\parskip}{5pt plus 2pt minus 1pt}

\begin{document}
\thispagestyle{firstpage}

\begin{logotype}
  {\includegraphics[scale=0.08]{USACHcolor.pdf}}
\end{logotype}

\vspace*{-1cm}

\begin{titulo}
Proyectos -- F�sica Computacional IV
\end{titulo}

\vspace*{0.2cm}

\begin{authorinfo}
  {Omar Fern�ndez Olgu�n}
  {omar.fernandez.o@usach.cl}
  {Nicol�s Campos Agusto}
  {nicolas.campos.a@usach.cl}
  {Astrof�sica con menci�n en ciencia de datos}
  {Departamento de F�sica, Facultad de Ciencias, Universidad de Santiago de Chile}
  {\today}
\end{authorinfo} 

\vspace*{0.2cm}
\vspace{1ex}\hrule\vspace{1ex}

\noindent \large{\textbf{\textsf{Migraci�n Orbital de Part�culas en Discos Protoplanetarios Huecos}}} \normalsize

\noindent En este proyecto se estudiar� la migraci�n de part�culas s�lidas inmersas en un disco protoplanetario que rodea a una estrella central. El disco presenta un \textbf{hueco interno}, es decir, no hay gas dentro de un radio \(r_\text{min}\) alrededor de la estrella. El gas ejerce dos efectos principales sobre las part�culas:

\begin{enumerate}
\item Su \textbf{gravedad}, que contribuye al potencial total del sistema.
\item Su \textbf{arrastre}, que tiende a acelerar o frenar a las part�culas hasta que se adapten a la velocidad local del gas.
\end{enumerate}

El objetivo es simular de forma num�rica c�mo estas part�culas evolucionan radialmente en el disco debido a estas interacciones.

\vspace{0.3cm}

\noindent \textbf{Etapas del proyecto:}

\begin{enumerate}

\item \textbf{C�lculo del potencial gravitatorio del disco y estrella}

\begin{itemize}
\item Considerar una densidad de masa radial del disco, expresada en funci�n del radio \(r = \sqrt{x^2 + y^2}\):
\[
\rho_\text{disk}(r) =
\begin{cases}
0 & r < r_\text{min},\\[1mm]
\rho_0 \left(\frac{r_0}{r}\right)^p & r \ge r_\text{min},
\end{cases}
\]
donde \(r_\text{min}\) define el hueco interno y \(p\) es un �ndice de densidad radial.
\item La estrella central se modela como un n�cleo compacto uniforme de radio \(R_\star\):
\[
\rho_\star(x,y) =
\begin{cases}
\dfrac{M_\star}{\pi R_\star^2}, & \sqrt{x^2+y^2} \le R_\star \\[1ex]
0, & \sqrt{x^2+y^2} > R_\star
\end{cases}
\]
\item Resolver la ecuaci�n de Poisson en 2D cartesianas considerando ambas contribuciones:
\[
\nabla^2 \Phi(x,y) = 4 \pi G \left[ \rho_\star(x,y) + \rho_\text{disk}(x,y) \right]
\]
usando diferencias finitas.
\item Se obtiene el potencial \(\Phi(x,y)\) en toda la malla.
\end{itemize}

\item \textbf{C�lculo del campo gravitatorio}

\begin{itemize}
\item El campo se obtiene mediante:
\[
\mathbf{g}(x,y) = - \nabla \Phi(x,y) = -\left(\frac{\partial \Phi}{\partial x}, \frac{\partial \Phi}{\partial y}\right)
\]
\item Discretizar el gradiente usando diferencias finitas centradas.
\item Este campo incluye tanto la gravedad de la estrella central como la del disco.
\end{itemize}

\item \textbf{Integraci�n de las trayectorias de las part�culas}

\begin{itemize}
\item Cada part�cula sigue la segunda ley de Newton con arrastre:
\[
\frac{d \mathbf{v}}{dt} = \mathbf{g}(x,y) - \gamma (\mathbf{v} - \mathbf{v}_\text{gas}(r)), \quad
\frac{d \mathbf{r}}{dt} = \mathbf{v}.
\]
\item El coeficiente de arrastre \(\gamma\) puede ser constante.
\item La velocidad del gas se define como un campo azimutal en funci�n de \(r\):
\[
\mathbf{v}_\text{gas}(r) = v_0 \left(\frac{r_0}{r}\right)^{\alpha} \hat{\boldsymbol{\phi}}, \quad
\hat{\boldsymbol{\phi}} = \left(-\frac{y}{r}, \frac{x}{r}\right)
\]
\item Integrar usando m�todos num�ricos como \code{Runge-Kutta 4} o \code{Verlet}.
\item Registrar la evoluci�n radial y las trayectorias de las part�culas.
\end{itemize}

\item \textbf{Visualizaci�n}

\begin{itemize}
\item Mapas del potencial \(\Phi(x,y)\) y del campo \(\mathbf{g}(x,y)\).
\item Trayectorias de part�culas mostrando la migraci�n radial en el disco.
\item Comparaci�n de diferentes perfiles de densidad del disco, radios del hueco interno y valores de arrastre.
\end{itemize}

\end{enumerate}

\noindent \textbf{Nota:} Este modelo simplificado no considera la presi�n del gas ni efectos hidrodin�micos complejos; el objetivo es centrarse en la interacci�n gravitatoria y el arrastre b�sico para estudiar la migraci�n de part�culas.

\end{document}
